\documentclass{uulm-assignment}

\usepackage{import}
\usepackage{tabularx}
\usepackage{listings}
\usepackage{hyperref}
\usepackage{CJKutf8}
\usepackage{graphicx}
\usepackage{verbatim} % wieder löschen später

\setboolean{showsolutions}{false}

\ifthenelse{\boolean{showsolutions}}{
\newcommand\mitloesung{1}%
}{
\newcommand\mitloesung{0}%
}


% Für Korrektur-Kommentare in roten Boxen:
\newcommand{\flo}[1]{
    \fcolorbox{purple}{pink}{\sffamily\scriptsize\bfseries\textcolor{black}{Flo:}} {\sffamily\bfseries\textcolor{purple}{#1}}
}



\hypersetup{colorlinks=false,urlcolor=uulm-in}

\faculty{Institut für Softwaretechnik und Programmiersprachen\hspace{0.05cm}}
\course{Softwaregrundprojekt}
\semester{\hspace{0.05cm}WiSe 2021/22}
\supervisor{\textbf{} \hspace{7.9cm} Prof. Dr. Thomas Thüm, Florian Ege, Dennis Jehle}

\assignmenttype{}
\assignmentno{}
\title{Pflichtenheft}

\begin{document}

\maketitle

\section{Einleitung}

Der Zweck dieses Dokuments ist es, eine detaillierte Beschreibung der Anforderungen, sowie der
Benutzerschnittstelle für die Anwendung \textbf{Dune TD} bereitzustellen. Es wird abgegrenzt, welche
Anforderungen erfüllt werden müssen, damit die entwickelte Anwendung vom Kunden akzeptiert
wird.

\subsection{Anwendungsbereich}

Das Spiel \textbf{Dune TD} ist eine Java-Anwendung, die es dem Benutzer ermöglicht, ein 3D-Spiel zu spielen.

\subsection{Definitionen und Abkürzungen}

Diese Tabelle enthält Abkürzungen und domänenspezifische Begriffe, die im Dokument verwendet
werden.

\begin{tabularx}{16cm}{l|X}
\textbf{Begriff} & \textbf{Definition} \\
\hline
Benutzer & In diesem Dokument wird immer dann von einem Benutzer gesprochen, wenn eine Person gemeint ist, die mit der Anwendung interagiert. Eine Person, die mit der Anwendung interagiert, ist solange ein Benutzer, bis diese sich innerhalb einer Spielpartie befindet, dann wird von einem Spieler gesprochen.\\
\hline
Spieler & Ein Spieler ist eine Person (männlich, weiblich, divers), welche sich innerhalb einer Spielpartie befindet und somit aktiv das Spiel Dune TD spielt. \\
\hline
Spielpartie & Unter einer Spielpartie versteht man die aktive Ausführung des Regelwerks des Spiels 'Dune TD', bis der Spieler eine Siegbedingung erreicht oder das Spiel beendet wird. \\
\hline
Client & Unter einem Client versteht man, im Kontext dieses Dokuments, eine Java Anwendung, welche das Spiel 'Dune TD' darstellt und dem Benutzer (Spieler) die Interaktion mit der Anwendung ermöglicht, sowie Nachrichten mit einem Server austauschen kann. \\
\hline
Server & Unter einem Server versteht man, im Kontext dieses Dokuments, eine Anwendung, die auf einem, über das Netzwerk erreichbaren, System installiert ist und Nachrichten mit mehreren Clients austauschen kann. \\
\end{tabularx}

\textbf{\textit{Hinweis: Dieser Abschnitt des Pflichtenhefts ist von hoher Relevanz. Das definieren von verwendeten Begriffen mag vielleicht zunächst überflüssig erscheinen, da Ihnen als Entwickler die Bedeutung der Begrifflichkeiten bekannt ist, allerdings ist das Pflichtenheft hauptsächlich ein Dokument für den Kunden. Pflegen Sie diesen Abschnitt des Pflichtenhefts daher mit großer Sorgfalt, denn es könnte durchaus vorkommen, dass sich Ihr Tutor / Ihre Tutorin in einen technisch nicht versierten Kunden verwandelt und den Rotstift schwingt.}}

\subsection{Überblick}

Der Rest dieses Dokuments enthält zwei Kapitel. Das zweite Kapitel gibt einen Überblick über die
Systemfunktionalität und behandelt die Systemeinschränkungen und Annahmen über das Produkt.
Das dritte Kapitel stellt die detaillierte Anforderungsspezifikation bereit.

\section{Allgemeine Beschreibung}

Dune TD ist ein rundenbasiertes Einzelspielerspiel aus dem Genre der Tower Defense Spiele (\href{https://de.wikipedia.org/wiki/Tower_Defense}{Link zu Wikipedia}). Es nimmt genau ein Spieler an einer Spielpartie teil. Ziel des Spiels ist es, alle Wellen von Feindeinheiten zu besiegen.

\subsection{Ansichten}

Die Anwendung besteht aus mehreren Ansichten, sogenannten \textit{Views}, über welche der Benutzer mit dem Programm interagieren kann.

\begin{tabularx}{16cm}{l|X}
\textbf{View} & \textbf{Beschreibung} \\
\hline
Hauptmenü & Nach dem Start der Anwendung befindet sich der Benutzer im Hauptmenü. Von hier aus kann der Benutzer auf verschiedene andere Ansichten navigieren. \\
\hline
Spielbildschirm & Auf dieser View wird das eigentliche Spiel dargestellt. \\
\end{tabularx}

\subsection{Systemeinschränkungen und Abhängigkeiten}

Die Anwendung wird durch die Prozessor- und/oder Grafikleistung des Systems begrenzt, auf dem es
läuft. Um die Anwendung auszuführen, wird das \textbf{Java Runtime Environment} (JRE) benötigt.

\section{Spezifische Anforderungen}

Dieser Abschnitt enthält alle spezifischen Anforderungen an das System. Er bietet eine detaillierte
Beschreibung des Systems und seiner Funktionen.

\subsection{Funktionale Anforderungen}

Dieser Abschnitt enthält alle Anforderungen, die die grundlegenden Aktionen des Softwaresystems
spezifizieren.

\begin{tabularx}{16cm}{l|X}
\textbf{ID} & \textbf{FA1} \\
\hline
TITEL: & Hauptmenü \\
\hline
BESCHREIBUNG: & Nach dem Anwendungsstart wird dem Benutzer das Hauptmenü angezeigt. Der Benutzer kann folgende Aktionen im Hauptmenü ausführen: 
\begin{itemize}
\item Auf den Spielbildschirm wechseln
\item Die Anwendung beenden
\end{itemize}
\\
\hline
BEGRÜNDUNG: & Nachdem der Benutzer die Anwendung gestartet hat, soll nicht direkt der Spielbildschirm erscheinen.
Der Benutzer soll die Möglichkeit haben das Spiel zu starten sobald es vom Benutzer gewünscht ist.\\
\hline
ABHÄNGIGKEITEN: & FA2\\
\end{tabularx}

\begin{tabularx}{16cm}{l|X}
\textbf{ID} & \textbf{FA2} \\
\hline
TITEL: & Spielbildschirm \\
\hline
BESCHREIBUNG: & Die Spieler bekommen auf dem Spielbildschirm den aktuellen Spielzustand angezeigt.
\\
\hline
BEGRÜNDUNG: & Es muss einen Bildschirm geben, auf welchem der Spielzustand angezeigt wird, da das Spiel
sonst nicht spielbar ist. \\
\hline
ABHÄNGIGKEITEN: & FA1, FA3\\
\end{tabularx}

\begin{tabularx}{16cm}{l|X}
\textbf{ID} & \textbf{FA3} \\
\hline
TITEL: & Spielfeld anzeigen \\
\hline
BESCHREIBUNG: & Das Spielfeld der Dune TD ist schachbrettartig, und hat eine Mindestgröße von 2 auf 2 Kacheln.
\\
\hline
BEGRÜNDUNG: & Es muss ein für die Spieler sichtbares Spielfeld geben, da ansonsten keine Türme
platziert werden können. \\
\hline
ABHÄNGIGKEITEN: & FA2\\
\end{tabularx}

\begin{tabularx}{16cm}{l|X}
	\textbf{ID} & \textbf{FA14} \\
	\hline
	TITEL: & Turmreichweite \\
	\hline
	BESCHREIBUNG: & Jeder Turm verfügt über einen effektiven Wirkbereich, dieser ist im Fall von \mbox{Geschütz-,} Bomben- und Schalltürmen kreisförmig.
	Im Fall des Shai-Hulud (Sandwurm) die komplette entsprechende Zeile oder Spalte des Spielfelds.\\
	\hline
	BEGRÜNDUNG: & Sonst könnte jeder Turm jede Feindeinheit von überall aus angreifen. \\
	\hline
	ABHÄNGIGKEITEN: & \\
\end{tabularx}

\begin{tabularx}{16cm}{l|X}
	\textbf{ID} & \textbf{FA15} \\
	\hline
	TITEL: & Turmreichweite: Geschütz- und Bombenturm \\
	\hline
	BESCHREIBUNG: & Der Geschütz- und Bombenturm attackieren Feindeinheiten aktiv, indem sie Projektile oder Bomben auf diese abfeuern.
	Doch nicht alle Feindeinheiten können angegriffen werden, nur solche welche sich innerhalb der Turmreichweite befinden.
	Um zu berechnen, ob sich eine Feindeinheit in Reichweite befindet wird die euklidische Norm verwendet (Turmposition, Position Feindeinheit).\\
	\hline
	BEGRÜNDUNG: & Sonst könnten Geschütz- und Bombentürme nicht entscheiden ob eine Feindeinheit beschossen werden kann oder nicht.\\
	\hline
	ABHÄNGIGKEITEN: & F14\\
\end{tabularx}

\begin{tabularx}{16cm}{l|X}
	\textbf{ID} & \textbf{FA16} \\
	\hline
	TITEL: & Turmreichweite: Schallturm \\
	\hline
	BESCHREIBUNG: & Der Schallturm als sogenannter Effekt-Turm greift Feindeinheiten nicht aktiv an, sondern verändert die Bodenbeschaffenheit in seinem Wirkbereich.
	Dennoch wird auch hier mittels der euklidischen Norm (Turmposition, Position Feindeinheit) ermittelt, ob Feindeinheiten vom Effekt des Schallturms betroffen sind oder nicht.\\
	\hline
	BEGRÜNDUNG: & Sonst könnten Feindeinheiten mit normaler Geschwindigkeit über Treibsand laufen.\\
	\hline
	ABHÄNGIGKEITEN: & F14\\
\end{tabularx}

\begin{tabularx}{16cm}{l|X}
\textbf{ID} & \textbf{FA36} \\
\hline
TITEL: & Score anzeigen \\
\hline
BESCHREIBUNG: & Der Score berechnet sich aus verschieden Faktoren. Durch das Zerstören von Feinden bekommt der Spieler Punkte, durch den Abriss von Türmen verliert er Punkte. Endet das Spiel durch einen Sieg oder durch eine Niederlage, ergeben sich die Finalen Punkte, welche mit den Multiplikatoren der Anzahl und Schwierigkeit der Wellen berechent werden.
\\
\hline
BEGRÜNDUNG: & So kann der Spieler seine aktuellen Punktestand einsehen.\\
\hline
ABHÄNGIGKEITEN: & FA2, FA53, FA54, FA56 \\
\end{tabularx}

\begin{tabularx}{16cm}{l|X}
\textbf{ID} & \textbf{FA37} \\
\hline
TITEL: & Highscore übermitteln \\
\hline
BESCHREIBUNG: & Nach dem Ende einer Spielpartie soll die erreichte Punktezahl an einen WebSocket Server übertragen werden.
\textbf{\textit{Hinweis: Die Nachrichten welche an den WebSocket Server gesendet werden müssen, um
die Punktezahl zu übermitteln werden im sogenannten Netzwerkstandarddokument spezifiziert, dieses Dokument
werden Sie rechtzeitig von uns erhalten. Für das Gruppenprojekt werden Sie selbst ein 
Netzwerkstandarddokument, zusammen mit anderen Studierenden, verfassen, dazu zu gegebener Zeit mehr.}}
\\
\hline
BEGRÜNDUNG: & So wird dokumentiert, welcher Spieler wann gegen wen gewonnen hat. \\
\hline
ABHÄNGIGKEITEN: & FA36, FA55, FA56 \\
\end{tabularx}

\begin{tabularx}{16cm}{l|X}
\textbf{ID} & \textbf{FA38} \\
\hline
TITEL: & Lebenspunkte anzeigen \\
\hline
BESCHREIBUNG: & Der Spieler verfügt über eine bestimmte Anzahl von Lebenspunkte. Diese werden reduziert, sobald eine Feindeinheit das Ausgangsportal des Spielfelds erreicht hat.
\\
\hline
BEGRÜNDUNG: & Es muss eine für den Spieler sichtbare Lebenspunkteanzeige geben, da der Spieler sonst nicht weiß, wie viele Feindeinheiten das Ausgangsportal noch erreichen könnten, bevor er das Spiel verloren hat \\
\hline
ABHÄNGIGKEITEN: &FA39, FA2 \\
\end{tabularx}

\begin{tabularx}{16cm}{l|X}
\textbf{ID} & \textbf{FA39} \\
\hline
TITEL: & Portale \\
\hline
BESCHREIBUNG: & Die Portale sind Kacheln auf dem Spielfeld an denen die Feindeinheiten das Spielfeld entweder betreten oder verlassen können. Zwischen diesen muss jedoch mindestens eine Kachel liegen. 
\\
\hline
BEGRÜNDUNG: & Ohne die Portale können die Feindeinheiten das Spielfeld nicht betreten oder verlassen. Somit könnte das Spiel nicht beendet werden. \\
\hline
ABHÄNGIGKEITEN: & FA2\\
\end{tabularx}

\begin{tabularx}{16cm}{l|X}
\textbf{ID} & \textbf{FA40} \\
\hline
TITEL: & Eingangsportal \\
\hline
BESCHREIBUNG: & Das Eingangsportal ist das Feld in dem die Gegner das Spielfeld beteten können. Nur dort können neue Gegner erscheinen. Zwischen dem Eingangs- und Ausgangsportal muss mindestens eine Kachel liegen.
\\
\hline
BEGRÜNDUNG: &  Sonst hätten die Gegner keinen Punkt an dem sie das Spielfeld betreten könnten\\
\hline
ABHÄNGIGKEITEN: &FA39 \\
\end{tabularx}

\begin{tabularx}{16cm}{l|X}
\textbf{ID} & \textbf{FA41} \\
\hline
TITEL: & Ausgangsportal \\
\hline
BESCHREIBUNG: & Das Ausgangsportal ist das Feld in dem die Gegner das Spielfeld wieder verlassen können. Sollten sie dieses erreichen, wird dem Spieler ein Lebenspunkt abgezogen. Zwischen dem Eingangs- und Ausgangsportal muss mindestens eine Kachel liegen.
\\
\hline
BEGRÜNDUNG: & So wird dokumentiert, welcher Spieler wann gegen wen gewonnen hat. \\
\hline
ABHÄNGIGKEITEN: & FA39\\ 
\end{tabularx}

\begin{tabularx}{16cm}{l|X}
\textbf{ID} & \textbf{FA42} \\
\hline
TITEL: & Teilnehmeranzahl \\ 
\hline
BESCHREIBUNG: & Dune TD darf nur von einem Spieler auf einem Client gleichzeitig gespielt werden. 
\\
\hline
BEGRÜNDUNG: & Da das Spiel als Einzelspieler-Spiel programmiert werden soll, darf maximal ein Spieler gleichzeitig auf das Spielfeld Zugriff haben. \\
\hline
ABHÄNGIGKEITEN: & \\
\end{tabularx}

\begin{tabularx}{16cm}{l|X}
\textbf{ID} & \textbf{FA43} \\
\hline
TITEL: & Platzierung von Türmen\\
\hline
BESCHREIBUNG: & Ein Turm wandelt eine Kachel von einem Laufweg zu einer Verteidigungseinheit um. Pro Kachel darf nur ein Turm platziert werden. Türme können nicht auf Portalen platziert werden. Dort kann sich dann keine Feindeinheit mehr befinden. Die Platzierung von Türmen kostet den Spieler Spice. Türme können nur vor der ersten und zwischen zwei Wellen platziert werden.
\\ 
\hline
BEGRÜNDUNG: & Ohne die Türme würde keine Möglichkeit bestehen die Feindeinheiten effektiv auf ihrem Weg zum Ausgangsportal aufzuhalten. \\ 
\hline
ABHÄNGIGKEITEN: & FA45, FA52\\
\end{tabularx}

\begin{tabularx}{16cm}{l|X}
\textbf{ID} & \textbf{FA44} \\
\hline
TITEL: & Anordnung von Türmen \\
\hline
BESCHREIBUNG: & Türme können nur auf einer Kachel gebaut werden. Sie dürfen jedoch nicht so platziert werden, dass die Türme den Weg der Feindeinheiten zum Ausgangsportal blockieren.
\\
\hline
BEGRÜNDUNG: & So wird gewährleistet, dass die Feindeinheiten eine Möglichkeit haben das Ausgangsportal zu erreichen. \\
\hline
ABHÄNGIGKEITEN: & FA43 \\ 
\end{tabularx}

\begin{tabularx}{16cm}{l|X}
\textbf{ID} & \textbf{FA45} \\
\hline
TITEL: & Spice anzeigen \\ 
\hline
BESCHREIBUNG: & Um Türme bauen zu können benötigt der Spieler Spice. Dieses kann er verdienen, indem er Feindeinheiten zerstört. Je nach Stärke der Feindeinheit bekommt er einen unterschiedlichen Betrag an Spice gutgeschrieben.
\\
\hline
BEGRÜNDUNG: & Da Türme nicht kostenlos gebaut werden können, muss der Spieler sehen, wie viel Spice ihm zur Verfügung steht. \\
\hline
ABHÄNGIGKEITEN: & FA2, FA47\\ 
\end{tabularx}

\begin{tabularx}{16cm}{l|X}
\textbf{ID} & \textbf{FA46} \\
\hline
TITEL: & Shai-Hulud \\
\hline
BESCHREIBUNG: & Der Shai-Hulud ist eine Verteidigungseinheit, welcher nach zwei Klopfern auf unterschiedlichen freien Kacheln in einer Reihe angelockt wird. Dieser "frisst" dann alles, was sich in dieser Reihe befindet, inklusive eigene Türme, exklusive Portale und Kacheln. Er kann in unregelmäßigen Abständen herbeigerufen werden, jedoch darf er maximal einmal pro Welle herbeigerufen werden.
\\
\hline
BEGRÜNDUNG: & Durch diese Einheit, können Feindeinheiten effektiv entfernt werden, ohne jeodch eine zu großen Vorteil zu bekommen, da eigene Türme damit auch zerstört werden. \\
\hline
ABHÄNGIGKEITEN: & FA52\\
\end{tabularx}

\begin{tabularx}{16cm}{l|X}
\textbf{ID} & \textbf{FA47} \\
\hline
TITEL: &Feindeinheiten \\
\hline
BESCHREIBUNG: & Die Feindeinheiten bestehen aus Infanterie, Erntemaschinen und Bosseinheiten, welche versuchen vom Startportal zum Endportal zu kommen, ohne dabei von den Türmen oder dem Shai-Hulud zerstört zu werden. Auf ihrem Weg werden sie von den Verteidigungseinheiten angegriffen, was sich durch eine Verringerung ihrer Lebenspunkte oder einem anderen Effekt auf sie auswirkt. 
\\
\hline
BEGRÜNDUNG: & Durch die Feindeinheiten bekommen die Verteidigungseinheiten (Türme, Shai-Hulud) eine Verwendung.\\
\hline
ABHÄNGIGKEITEN: & \\
\end{tabularx}

\begin{tabularx}{16cm}{l|X}
\textbf{ID} & \textbf{FA48} \\
\hline
TITEL: & Feindeinheiten: Infanterie \\
\hline
BESCHREIBUNG: & Die Infanterie ist die schnellste Feindeinheit. Diese ist jedoch auch die am wenigsten widerstandsfähigste. Die effektivste Verteidigungseinheit gegen sie ist der Geschützturm.
\\
\hline
BEGRÜNDUNG: & Durch diese Einheit, muss der Spieler schnell und taktisch handeln, da diese schwer durch bspw. einen Bombenturm aufgehalten werden können. \\
\hline
ABHÄNGIGKEITEN: & FA47\\ 
\end{tabularx}

\begin{tabularx}{16cm}{l|X}
\textbf{ID} & \textbf{FA49} \\
\hline
TITEL: & Feindeinheiten: Erntemaschinen \\
\hline
BESCHREIBUNG: & Die Erntemaschinen sind die langsamsten Einheiten. Diese haben dafür sehr hohe Lebenspunkte. Die effektivste Verteidigungseinheit gegen sie ist der Bombenturm.
\\
\hline
BEGRÜNDUNG: & Durch diese Einheit, muss der Spieler seien Türme schlau platzieren, da diese schwer durch bspw. einen Geschützturm oder einen Schallturm aufgehalten werden können. \\ 
\hline
ABHÄNGIGKEITEN: & FA47\\ 
\end{tabularx}

\begin{tabularx}{16cm}{l|X}
\textbf{ID} & \textbf{FA50} \\
\hline
TITEL: & Feindeinheiten: Bosseinheiten  \\
\hline
BESCHREIBUNG: & Die Bosseinheit hat am meisten Lebenspunkte von allen Einheiten. Sie ist deutlich schneller als die Erntemaschine, jedoch nicht so schnell wie die Infanterie-Einheiten.
\\
\hline
BEGRÜNDUNG: & Durch diese Einheit wird das Spiel für den Spieler schweiriger gestaltet, da die Bosseinheit eine Mischung aus den beiden anderen Einheiten ist. \\ 
\hline
ABHÄNGIGKEITEN: & FA47\\
\end{tabularx}

\begin{tabularx}{16cm}{l|X}
\textbf{ID} & \textbf{FA51} \\
\hline
TITEL: & Feindeinheiten: Pathfinding \\ 
\hline
BESCHREIBUNG: & Die Feindeinheiten können sich nach links, rechts, oben und unten bewegen, jedoch nicht Diagonal. Die Feindeinheiten suchen sich immer den kürzesten Weg zum Endportal, was aber nicht bedeutet, dass dies der am wenigsten schädlichste ist. Es muss immer midenstens ein Laufweg gewährleistet sein. Um zu berechnen, welches der kürzeste Weg ist wird der  Dijkstra-Algorithmus verwendet. Sollten mehrere kürzeste Laufwege zur Verfügung stehen, wird ein Laufweg in determinischtischer Art und Weise gewählt.
\\ 
\hline
BEGRÜNDUNG: & Somit kann der Spieler evtl. vorhersehen, wie er seine Türme platzieren muss um das Spiel zu gewinnen und in einem großen Spielfeld können sich die Feindeinheiten nicht wirklich "verlaufen". \\ 
\hline
ABHÄNGIGKEITEN: & FA43, FA47\\ 
\end{tabularx}

\begin{tabularx}{16cm}{l|X}
\textbf{ID} & \textbf{FA52} \\
\hline
TITEL: & Wellen \\
\hline
BESCHREIBUNG: & Dass nicht alle Feindeinheiten auf einmal das Spielfeld betreten und der Spieler auch Zeit hat, um neune Verteidigungseinheiten zu platzieren, werden die Feindeinheiten in Wellen versuchen das Endportal zu erreichen. 
\\ 
\hline
BEGRÜNDUNG: & Durch die Wellen wird der Spielablauf strukturierter und der Spieler kann seine verdinenten Spice zwischen den Wellen einsetzen. \\
\hline
ABHÄNGIGKEITEN: & \\
\end{tabularx}

\begin{tabularx}{16cm}{l|X}
\textbf{ID} & \textbf{FA53} \\
\hline
TITEL: & Wellen: Anzahl \\ 
\hline
BESCHREIBUNG: & Das die Spiele nicht alle ewig gehen, kann der Spieler die Anzahl der Wellen vor dem Spiel definieren. Hierbei kann er entscheiden, ob er viele Wellen haben will, oder mit wenigen Wellen spielen will.
\\
\hline
BEGRÜNDUNG: & So kann der Spieler festlegen, wie lange sein Spiel gehen soll. \\
\hline
ABHÄNGIGKEITEN: &FA52 \\ 
\end{tabularx}

\begin{tabularx}{16cm}{l|X}
\textbf{ID} & \textbf{FA54} \\
\hline
TITEL: & Wellen: Schwierigkeit \\ 
\hline
BESCHREIBUNG: & Da nicht jeder Spieler gleich gut ist, kann er die Schwierigkeit der Wellen vor dem Spiel definieren. Somit wird die Anzahl der Feindeinheiten, sowie die Anzahl der verschiedenen Arten der Feindeinheiten in Bezug auf die Anzahl der Wellen (Wellen in aufsteigender Schwierigkeit) angepasst.
\\
\hline
BEGRÜNDUNG: & So kann der Spieler festlegen, wie schwer sein Spiel werden soll.\\
\hline
ABHÄNGIGKEITEN: & FA52, FA53\\ 
\end{tabularx}

\begin{tabularx}{16cm}{l|X}
\textbf{ID} & \textbf{FA55} \\
\hline
TITEL: & Sieg \\
\hline
BESCHREIBUNG: & Sollte der Spieler alle Feindeinheiten in allen Wellen besiegt haben, ohne dass seine Lebenspunkte auf 0 sinken, hat der Spieler das Spiel gewonnen.
\\
\hline
BEGRÜNDUNG: & So wird erkannt, dass der Spieler alle Wellen abgeschlossen hat. \\
\hline
ABHÄNGIGKEITEN: & FA52\\
\end{tabularx}

\begin{tabularx}{16cm}{l|X}
\textbf{ID} & \textbf{FA56} \\
\hline
TITEL: & Niederlage \\
\hline
BESCHREIBUNG: & Wenn der Spieler alle Lebenspunkte durch Feineinheiten, welche das Ausgangsportal erreicht haben, verloren hat, hat der Spieler das Spiel verloren.
\\
\hline
BEGRÜNDUNG: & So wird erkannt, dass die Lebenspunktanzalh auf 0 gesunken ist. \\
\hline
ABHÄNGIGKEITEN: & FA38\\
\end{tabularx}

\subsection{Nichtfunktionale Anforderungen}

Dieser Abschnitt spezifiziert die Qualitätsanforderungen (QA) an das Softwaresystem.

\begin{tabularx}{16cm}{l|X}
\textbf{ID} & \textbf{QA1} \\
\hline
TITEL: & Robustheit \\
\hline
BESCHREIBUNG: & Die Anwendung darf nicht abstürzen. Bei 100 Spielen darf maximal 1 Spiel
aufgrund eines Fehlers abgebrochen werden. \\ 
\end{tabularx}

\begin{tabularx}{16cm}{l|X}
\textbf{ID} & \textbf{QA2} \\
\hline
TITEL: &  Kompabilität\\ 
\hline
BESCHREIBUNG: & Die Anwendung muss auf Windows 10 und Linux laufähig sein. \\ 
\end{tabularx}

\begin{tabularx}{16cm}{l|X}
\textbf{ID} & \textbf{QA3} \\
\hline
TITEL: & Anwendungssprache \\ 
\hline
BESCHREIBUNG: & Die Anwendung muss in den Sprachen deutsch oder englisch ausführbar sein.  \\ 
\end{tabularx}

\end{document}
